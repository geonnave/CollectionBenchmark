\section{Sobre os Benchmarks} \label{sec:bench}
	Na computação são necessários testes de hardware e software para gerenciamento de sua performance, esses testes são conhecidos como benchmarks. Benchmark é o ato de executar um programa de computador, um conjunto de programas ou outras operações, a fim de avaliar a performance relativa de um objeto, normalmente executando uma série de testes padrões e ensaios nele.

	Benchmark é útil para o entendimento de como o gerenciador de banco de dados responde sob a variação de condições. Pode-se criar cenários que testam o tratamento de deadlock, performance dos utilitários, diferentes métodos de carregar dados, características da taxa de transição quando mais usuários são adicionados e ainda o efeito na aplicação usando uma nova versão do produto.
